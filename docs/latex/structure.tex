\section{Struttura della progetto}
L’implementazione del progetto è stata organizzata in modo modulare, seguendo una separazione tra la \textbf{libreria} sviluppata e i \textbf{programmi eseguibili} che ne fa uso. Di seguito si descrivono le principali componenti.

\begin{footnotesize}
\begin{verbatim}
jpeg-converter
├── cli
│   ├── __init__.py
│   └── cli.py
├── converter
│   ├── __init__.py
│   ├── backend.py
│   └── dct.py
├── gui
│   ├── __init__.py
│   ├── gui.py
│   └── resources
├── images
│   ├── 160x160.bmp
│   └── ...
├── main.py
├── pyproject.toml
└── tests
    ├── __init__.py
    ├── benchmark.py
    ├── parameters.py
    └── test.py
\end{verbatim}
\end{footnotesize}

\subsection{Panoramica generale}

La root della repository contiene tre directory principali:

\begin{itemize}
    \item \texttt{converter/}: libreria contenente l’implementazione della logica della Discrete Cosine Transform e le funzioni che permettono di applicarla ad un immagine.
    \item \texttt{cli/}: file contenenti la logica per avviare la compressione da linea di comando
    \item \texttt{gui/}: file contenenti il necessario per avviare l'interfaccia grafica
    \item \texttt{tests/}: file contenenti test e benchmark, spiegati di seguito.
\end{itemize}

Completano il progetto i file \texttt{pyproject.toml} e \texttt{requirements.txt}, che definiscono le dipendenze del progetto e script di esecuuzione.

\subsection{Verifica della correttezza}

Per verificare il corretto funzionamento dell'implementazione della DCT (Discrete Cosine Transform), sono stati eseguiti due test.\\

È stato applicato l'algoritmo della DCT monodimensionale a un vettore di 8 valori:
\begin{center}
\texttt{[231, 32, 233, 161, 24, 71, 140, 245]}
\end{center}
Il risultato è stato confrontato con una trasformata attesa nota:
\begin{center}
\texttt{[4.01e+02, 6.60e+00, 1.09e+02, -1.12e+02, 6.54e+01, 1.21e+02, 1.16e+02, 2.88e+01]}
\end{center}

Analogamente, è stato considerato un blocco $8 \times 8$ di valori in scala di grigi. L'algoritmo DCT2 è stato applicato al blocco e i risultati ottenuti sono stati confrontati con i coefficienti attesi forniti nel testo del progetto.

\subsubsection*{Criterio di verifica}
Il confronto tra i risultati ottenuti e quelli attesi è stato effettuato in termini di errore relativo massimo, confrontando elemento per elemento le due matrici. Il test è considerato superato se tutti i valori presentano un errore relativo inferiore a una soglia prefissata $\varepsilon = 10^{-2}$.


\subsection{Considerazioni}

La struttura modulare del progetto garantisce una \textbf{buona separazione delle responsabilità}, facilita l’estensione e favorisce la riusabilità della libreria in altri contesti. Inoltre il convertitore può essere usato sia da linea di comando in modo rapido e veloce, oppure da interfaccia grafica, rendendolo versatile in diversi contesti.