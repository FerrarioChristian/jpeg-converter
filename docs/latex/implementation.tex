\section{Implementazione della DCT e DCT2 personalizzata}

Per la compressione jpeg delle immagini abbiamo realizzato una \textbf{libreria personalizzata per il calcolo della DCT} (monodimensionale e bidimensionale), basata sulla definizione teorica della trasformata discreta del coseno. L'implementazione non utilizza algoritmi ottimizzati (come quelli basati su FFT), ma si fonda sull'utilizzo esplicito di una matrice di base $D$ contenente le funzioni coseno ed i suoi coefficienti.

\subsection{Costruzione della base (matrice $D$)}

Per un vettore di lunghezza $N$, la trasformata discreta del coseno può essere espressa come un prodotto matrice-vettore, dove la matrice $D \in \mathbb{R}^{N \times N}$ è definita da:

\[
D_{k,j} = \alpha(k) \cdot \cos\left( \frac{\pi}{N} \left(j + \frac{1}{2} \right) k \right) \quad \text{per } k,j = 0, \dots, N-1
\]

con:

\[
\alpha(k) = 
\begin{cases}
\sqrt{\frac{1}{N}} & \text{se } k = 0 \\
\sqrt{\frac{2}{N}} & \text{se } k > 0
\end{cases}
\]

Questa costruzione rende $D$ una matrice ortogonale, per cui l'inversa coincide con la trasposta: $D^{-1} = D^\top$.
\begin{footnotesize}
\begin{verbatim}
def dct_base(n):
    D = np.zeros((n, n))
    alpha = alphas(n)

    for i in range(n):
        for j in range(n):
            D[i, j] = alpha[i] * cos((i * pi * (2 * j + 1) / (2 * n)))
    return D

def alphas(n):
    alpha = np.zeros(n)
    alpha[0] = 1 / sqrt(n)
    for i in range(1, n):
        alpha[i] = sqrt(2 / n)
    return alpha
\end{verbatim}
\end{footnotesize}

\subsection{Implementazione della DCT monodimensionale}

Utilizzando la matrice $D$, la trasformata discreta del coseno (DCT) e la sua inversa (IDCT) si implementano nel modo seguente:

\begin{verbatim}
def dct(f, D):
    return D @ f

def idct(c, D):
    return D.T @ c
\end{verbatim}

Dove \texttt{f} è il vettore delle sequenze in ingresso e \texttt{c} il vettore dei coefficienti.

\subsection{Implementazione della DCT bidimensionale (DCT2)}

Nel caso bidimensionale, la DCT2 si ottiene applicando la trasformata una volta per riga e una volta per colonna. L'implementazione sfrutta nuovamente la matrice $D$ come segue:

\begin{verbatim}
def dct2(f, D):
    return D @ f @ D.T

def idct2(c, D):
    return D.T @ c @ D
\end{verbatim}

Dove \texttt{f} rappresenta un blocco di immagine $F \times F$ (o l'immagine stessa), e \texttt{c} è la matrice dei coefficienti trasformati.

\subsection{Eliminazione delle alte frequenze}

Per simulare un meccanismo di compressione simile a quello impiegato nello standard JPEG, è stato implementato un filtro che elimina le alte frequenze nella matrice dei coefficienti ottenuta dalla trasformata DCT2. 

L'eliminazione viene effettuata in ciascun blocco $F \times F$ dell'immagine trasformata, applicando una maschera diagonale binaria. Questa maschera ha valore 1 nei coefficienti per cui $i + j < D$ (dove $i$ e $j$ sono gli indici di riga e colonna), e 0 altrove. In questo modo, vengono mantenuti solo i coefficienti a bassa frequenza, che rappresentano la parte più significativa dell'informazione visiva, mentre vengono annullati quelli a più alta frequenza, che contribuiscono in misura minore alla qualità percepita.

La funzione \texttt{cut\_frequencies} applica tale maschera a ciascun blocco dell'immagine, riducendo efficacemente la quantità di dati senza alterare eccessivamente il contenuto visivo rilevante.
